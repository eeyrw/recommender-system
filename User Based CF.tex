\documentclass[12pt,a4paper]{ctexart}
\usepackage[utf8]{inputenc}
\usepackage{amsmath}
\usepackage{amsfonts}
\usepackage{amssymb}
\usepackage{graphicx}
\usepackage{bm}
\usepackage{booktabs}
%\usepackage{draftwatermark}
\usepackage[backend=biber,backref=true%nature,% citestyle=gb7714−2015,backref=true%
]{biblatex}
%参考文献数据源加载 
\addbibresource[location=local]{ref.bib}
\usepackage[table,xcdraw]{xcolor}
\usepackage{float}
%\usepackage[left=3.0cm, right=3.0cm, top=3.5cm, bottom=2.70cm]{geometry}
\title{协同过滤算法实验}
\author{Yuan}
\date{\small\today}
\begin{document}

\maketitle
\begin{abstract}
本文参考文献\Autocite{Herlocker:1999:AFP:312624.312682}中的关于User Based Collaborative Filtering的方法对基于用户相似度的协同过滤的算法进行了实现,并使用MovieLens\Autocite{moivelens1m_dataset}和豆瓣电影评分数据集\Autocite{douban_dataset}进行了实验。文章将从实验的背景介绍、实验数据集来源、实验方案、实验结果和分析这几个方面展开。
\end{abstract}	
\section{背景介绍}
协同过滤(Collaborative Filtering)作为推荐算法中最经典的类型,其模型一般为m个物品,n个用户的数据,只有部分用户和部分数据之间是有评分数据的,其它部分评分是空白,此时我们要用已有的部分稀疏数据来预测那些空白的物品和数据之间的评分关系,找到最高评分的物品推荐给用户。

一般来说,协同过滤推荐分为基于用户(user-based)的协同过滤和基于项目(item-based)的协同过滤,本文实现的是基于用户协同过滤的算法。基于用户(user-based)的协同过滤主要考虑的是用户和用户之间的相似度,只要找出相似用户喜欢的物品,并预测目标用户对对应物品的评分,就可以找到评分最高的若干个物品推荐给用户\Autocite{推荐系统实践}。而基于项目(item-based)的协同过滤和基于用户的协同过滤类似,只不过这时我们转向找到物品和物品之间的相似度,只有找到了目标用户对某些物品的评分,那么我们就可以对相似度高的类似物品进行预测,将评分最高的若干个相似物品推荐给用户。例如用户在网上买了一本统计学相关的书,网站马上会推荐一堆统计学,随机过程相关的书给用户,这就用到了基于项目的协同过滤思想。

\section{实验}
\subsection{实验所用数据集说明}
实验分别使用MoiveLens-1M数据集以及豆瓣电影评分数据集。数据集的统计信息如下所述:
% Please add the following required packages to your document preamble:
% \usepackage{booktabs}
\begin{table}[H]
	\centering
	\begin{tabular}{@{}llll@{}}
		\toprule
		数据集名称        & 评分样本数   & 电影总数  & 用户总数  \\ \midrule
		MoiveLens-1M & 1000209 & 3664  & 12080 \\
		豆瓣电影评分数据集    & 199813  & 22979 & 26237 \\ \bottomrule
	\end{tabular}
	\caption{数据集统计信息}
\label{table:dataset}
\end{table}
其中豆瓣电影评分数据集源自热心网友使用爬虫爬取豆瓣网并整理而得,MoiveLens-1M数据集为祖传数据集。豆瓣数据集较MoiveLens-1M数据集更为稀疏,能更好的验证用户相似度计算这一方法对于稀疏度无能为力的特性。


\subsection{实验方案}
由于评估性能的需要,数据集会被分割为测试集及训练集。这里采用训练样本集以0.7的概率被选中,测试集以0.3的概率被选中的方法随机从样本集合里抽取,并且采用不放回的方式抽取。实验基于用户相似度进行推荐的算法开展,使用Python3编写相关代码。
核心推荐系统的整体运作流程如下:
\begin{enumerate}
	\item 读入数据集样本,按照一定比例随机将数据集分为训练集和测试集。
	\item 分别得到对应集合的\{用户:\{电影1:评分,电影2:评分,$\cdots$\}\}的字典。
	\item 构建\{电影:[用户1,用户2,$\cdots$]\}的倒排字典。
	\item 使用倒排字典构建用户相关字典\{用户:\{相似用户1:同时评论的电影数目,相似用户2:同时评论的电影数目,$\cdots$\}\}。
	\item 使用用户相关字典计算用户余弦相似度并记录到用户相似度矩阵。
	\item 使用相似度矩阵对待推荐用户的周围K个相似用户推荐N部电影。
		
\end{enumerate}


实验评估采用准确率、召回率及覆盖度这三个评价指标。定义如下:
\[ \mbox{准确率}=\frac{\sum_{u \in U}|R(u) \bigcap T(u)|}{\sum_{u \in U}|R(u)|} \]
\[ \mbox{召回率}=\frac{\sum_{u \in U}|R(u) \bigcap T(u)|}{\sum_{u \in U}|T(u)|} \]
\[ \mbox{覆盖度}=\frac{\left|\bigcup_{u \in U} R(u)\right|}{|I|} \]
其中$U$为用户集合,$R(u)$为根据用户在训练集上的行为给出的用户做出的推荐列表,$T(u)$为用户在测试集上的行为列表。$I$为所有电影的集合。
\subsection{实验结果}
在两个不同数据集的实验结果如下表所示:
% Please add the following required packages to your document preamble:
% \usepackage{booktabs}
\begin{table}[H]
		\centering
	\begin{tabular}{@{}llll@{}}
		\toprule
		数据集          & 准确率    & 召回率    & 覆盖度    \\ \midrule
		豆瓣电影评分数据集    & 0.0001 & 0.0003 & 0.7929 \\
		MoiveLens-1M & 0.3756 & 0.0758 & 0.3204 \\ \bottomrule
	\end{tabular}
	\caption{实验结果}
\label{table:exp_result}
\end{table}
从结果可以发现豆瓣电影评分数据集的准确率和召回率都出奇的低,这是数据集过于稀疏所致。因此这也印证了利用相似度计算的推荐系统无法解决稀疏性问题。在MoiveLens-1M这样不是那么稀疏的数据集上能有近30\%的准确率,效果也并不是很理想。但是就覆盖度而言在豆瓣数据集上表现得更好,因为豆瓣数据集的电影数相比MovieLens的多,而且几乎和用户数量一样多。




\section{总结}
实验选择了两组不同的数据集,都使用基于用户的协同过滤算法进行验证。得到了不同结果,验证了使用用户相似度计算的协同过滤算法的缺陷。


\printbibliography[heading=bibliography,title=参考文献]
\end{document}